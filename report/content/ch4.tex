\chapter{项目实现}

\section{整体架构设计}
本项目采用模块化设计,主要分为以下几个模块:\par
\vspace{1em}

\begin{enumerate}
    \item \textbf{主程序模块}(main.py):程序入口,负责初始化界面和检查权限
    \item \textbf{嗅探器核心模块}(sniffer.py):实现数据包捕获、界面交互等核心功能
    \item \textbf{数据包处理模块}(process\_packet.py):负责解析不同协议的数据包
\end{enumerate}

\section{核心功能实现}

\subsection{主程序模块(main.py)}
主程序模块负责初始化应用程序,检查运行权限,并创建主窗口。

\begin{codeblock}[language=python]
import os
import tkinter as tk
from sniffer import NetworkSniffer

def main():
    """主函数"""
    # 检查是否以管理员/root权限运行
    if os.name == "nt":  # Windows
        try:
            import ctypes
            is_admin = ctypes.windll.shell32.IsUserAnAdmin() != 0
        except:
            pass
    else:  # Unix/Linux
        is_admin = (os.geteuid() == 0)

    root = tk.Tk()
    app = NetworkSniffer(root, is_admin)
    
    # 设置程序退出处理
    def on_closing():
        if app.is_sniffing:
            app._pause_sniffing()
        root.destroy()
    
    root.protocol("WM_DELETE_WINDOW", on_closing)
    root.mainloop()

if __name__ == "__main__":
    main()
\end{codeblock}

\subsection{嗅探器核心模块(sniffer.py)}
该模块实现了网络嗅探器的核心功能,包括基于Tkinter的界面创建、基于psutil的网络接口读取、基于scapy的数据包捕获、PCAP文件处理以及状态控制等。

\subsubsection{初始化与界面创建}
这一部分完成嗅探器初始化与Tkinter界面创建工作,同时基于psutil加载网络接口。

\begin{codeblock}[language=python]
class NetworkSniffer:
    """网络嗅探器主类"""
    
    def __init__(self, root: tk.Tk, is_admin):
        """初始化嗅探器"""
        self.root = root
        if is_admin:
            self.root.title("网络嗅探器(管理员模式)")
        else:
            self.root.title("网络嗅探器(建议打开管理员模式以体验完整功能)")
        self.root.geometry("1200x800")
        self.root.minsize(1200, 800)
        self.fonts = ("SimHei", 10)
        
        # 程序状态变量
        self.is_sniffing = False
        self.sniff_thread = None
        self.packet_list: List[Dict[str, Any]] = []
        self.original_packet_list: List[Dict[str, Any]] = []
        self.selected_packet_index = -1
        self.protocol_filters = {}
        self.current_interface = None
        self.packets_to_save = []
        self.is_loading_from_pcap = False
        self.is_filtered = False
        self.is_promiscuous = False
        self.is_admin = is_admin
        
        # 初始化界面
        self._create_widgets()
        self._layout_widgets()
        self._setup_events()
        
        # 加载网络接口
        self.load_network_interfaces()
\end{codeblock}

\subsubsection{数据包捕获功能}
这一部分负责数据包的捕获工作,通过调用Scapy库的接口实现。

\begin{codeblock}[language=python]
def _start_sniffing(self):
    """开始嗅探数据包"""
    # 省略相关处理过程
    self.sniff_thread = threading.Thread(target=self._sniff_packets, args=(interface, protocol_filter), daemon=True)
    self.sniff_thread.start()

def _sniff_packets(self, interface: str, protocol_filter: str):
    """数据包嗅探函数"""
    def packet_callback(packet):
        if not self.is_sniffing:
            return False
            
        self._process_packet(packet)
    
    # 构建过滤器表达式
    filter_expr = ""
    if protocol_filter != "全部":
        if protocol_filter == "IPv4":
            filter_expr = "ip"
        elif protocol_filter == "IPv6":
            filter_expr = "ip6"
        else:
            filter_expr = protocol_filter.lower()
    
    try:
        scapy.sniff(iface=interface, prn=packet_callback, store=False, stop_filter=lambda _: not self.is_sniffing, filter=filter_expr, promisc=self.is_promiscuous)  # 根据标志启用/禁用混杂模式
    except Exception as e:
        self.root.after(0, lambda: messagebox.showerror("错误", f"嗅探失败: {str(e)}"))
        self.root.after(0, self._pause_sniffing)
\end{codeblock}

\subsubsection{PCAP文件处理}
对于PCAP文件的处理,可以调用Scapy库提供的接口wrpcap、rdpcap快速实现。

\begin{codeblock}[language=python]
def save_to_pcap(self):
    """保存为PCAP文件"""        
    try:
        # 打开文件对话框
        file_path = filedialog.asksaveasfilename(
            defaultextension=".pcap",
            filetypes=[("PCAP Files", "*.pcap"), ("All Files", "*")]
        )
        
        if file_path:
            # 保存数据包
            wrpcap(file_path, self.packets_to_save)
            messagebox.showinfo("成功", f"数据包已保存到 {file_path}")
    except Exception as e:
        messagebox.showerror("错误", f"保存失败: {str(e)}")

def load_from_pcap(self):
    """从PCAP文件加载数据包(使用线程避免GUI卡死)"""
    try:
        # 打开文件对话框
        file_path = filedialog.askopenfilename(
            filetypes=[("PCAP Files", "*.pcap"), ("All Files", "*")]
        )
        
        if file_path:
            # 启动加载线程
            load_thread = threading.Thread(target=_load_packets_thread, daemon=True)
            load_thread.start()
            
    except Exception as e:
        messagebox.showerror("错误", f"加载失败: {str(e)}")
        # 重置加载标志
        self.is_loading_from_pcap = False
\end{codeblock}

\subsubsection{其他操作}
在嗅探器核心模块中,我还实现了界面更新、清空、数据包筛选、数据包处理调用等操作,由于不是重点,因此不多赘述。


\subsection{数据包处理模块(process\_packet.py)}
该模块负责解析不同协议的数据包,基于python scapy库提供的接口提取各种关键信息,处理后添加到\texttt{packet\_info}字典中,返回给\texttt{NetworkSniffer}类实例进行处理并展示。

\subsubsection{IP协议处理}
以网络层的IP协议为例,下面是具体的处理逻辑:

对于Scapy库捕获的原始网络数据包对象,先校验是否含 IP 层,无则直接返回。接着提取 IP 对象后,通过协议编号映射表获取上层协议名。然后,根据IP数据包首部格式判断是否为分片包:若DF位为0(表示支持分片),且MF位为1(表示后续存在分片)或偏移不为0,则是分片数据包。接着存入协议类型、源 / 目的 IP、TTL 等核心信息,拼接摘要字符串。同时填充 IPv4 版本、头部长度(ihl*4)、服务类型、总长度、标识等详细字段,最后若有 Raw 层则提取应用层原始数据,为网络流量分析提供结构化 IP 层信息。

\begin{codeblock}[language=python]
def process_ip(packet: scapy.Packet, packet_info: Dict[str, Any]):
    """处理IP层"""
    if IP not in packet:
        return

    ip_packet = packet[IP]
    # 获取协议类型名称
    IP_proto_names = {
        1: "ICMP",
        2: "IGMP",
        6: "TCP",
        17: "UDP",
        # 其他协议...
    }
    IP_proto_name = IP_proto_names.get(ip_packet.proto, f"Protocol {ip_packet.proto}")

    # 检查是否是分片数据包
    is_fragment = (not ip_packet.flags.DF) and (ip_packet.flags.MF or ip_packet.frag > 0)
    if is_fragment:
        packet_info["is_fragment"] = "1"
    
    packet_info["protocol"] = "IP"
    packet_info["info"] = f"IP {ip_packet.src} -> {ip_packet.dst} TTL={ip_packet.ttl}"
    
    packet_info["src"] = ip_packet.src
    packet_info["dst"] = ip_packet.dst
    
    # 详细信息
    packet_info["detail"] += f"=== IPv4 层 ===\n"
    packet_info["detail"] += f"版本: {ip_packet.version}\n"
    packet_info["detail"] += f"头部长度: {ip_packet.ihl * 4} 字节\n"
    packet_info["detail"] += f"服务类型: {ip_packet.tos}\n"
    packet_info["detail"] += f"数据包总长度: {ip_packet.len} 字节\n"
    packet_info["detail"] += f"数据包标识: {ip_packet.id}\n"
    # 其他详细信息提取

    # 尝试提取应用层数据
    if packet.haslayer(scapy.Raw):
        packet_info["data"] = packet[scapy.Raw].load
\end{codeblock}

\subsubsection{其他协议处理}
类似地,模块中还实现了对TCP、UDP、ICMP、ARP等协议的解析处理函数,这里不再一一展示。

\subsubsection{分片重组处理}
IP数据包分片的重组处理主要依赖于两个函数,\texttt{reassemble\_packet}实现重组功能,并返回给\texttt{NetworkSniffer}类的成员函数进行处理并展示。判断分片数据包的方式与IP协议处理部分类似,这里不过多赘述。

\begin{codeblock}[language=python]
def reassemble_packet(Sniffer):
    """重组分片数据包"""
        
    # 获取当前选中的IP数据包信息,收集具有相同IP ID、源IP、目的IP和协议的所有分片及其对应的packet_info加入fragments_with_info,并根据偏移量对其进行排序
    
    # 尝试重组数据包
    try:
        reassembled_data = b""

        for _, fragment, _, packet_info in fragments_with_info:
            # 优先使用packet_info中已解析的数据
            if packet_info["data"]:
                fragment_data = packet_info["data"]
            else:
                # 如果没有预解析数据,直接从IP包中提取载荷
                ip_header_length = fragment.ihl * 4
                fragment_data = bytes(fragment)[ip_header_length:]
            # 将当前分片数据添加到重组数据中
            reassembled_data += fragment_data
            
        # 检查分片完整性、从偏移量0开始的连续性、中间分片的连续性、是否有最后一个分片(没有MF标志)
        
        reassembled_detail = "=== 重组结果 ===\n"
        reassembled_detail += f"IP ID(数据报标识): {ip_id}\n"
        reassembled_detail += f"源IP: {src_ip}\n"
        reassembled_detail += f"目的IP: {dst_ip}\n"
        reassembled_detail += f"协议: {proto}\n"
        reassembled_detail += f"数据长度: {max_offset} 字节\n"
        reassembled_detail += f"是否完整: {'是' if is_complete else '否'}\n"
        reassembled_detail += f"缺失分片: {', '.join(missing_fragments) if not is_complete else '无'}\n\n"
        # 其余详细信息
        return reassembled_detail, reassembled_data 

    except Exception as e:
        messagebox.showerror("重组错误", f"重组过程中发生错误: {str(e)}")
        return None, None
\end{codeblock}


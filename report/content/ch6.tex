\chapter{总结}

\section{项目成果}
本项目成功实现了一个功能完善的网络嗅探器,具备以下特点:

\begin{enumerate}
    \item 支持多种网络协议的解析,包括TCP、UDP、ICMP、ICMPv6、IPv4、IPv6、ARP等
    \item 提供直观友好的图形界面,方便用户操作和数据分析
    \item 实现了数据包过滤、分片重组、混杂模式等高级功能
    \item 具备良好的可扩展性,便于后续添加新的协议解析和功能扩展
\end{enumerate}

\section{项目不足与改进方向}
由于时间有限,本项目还存在许多不足之处,如:对于一些不常见的网络协议支持不够完善;在处理大量数据包时,即便使用多线程并对数据包进行分批处理,界面可能出现卡顿现象;数据包分析功能相对基础,缺乏深度分析能力等等。

\vspace{1em}
如若后续有更多时间,我会考虑从以下方面对项目进行进一步的优化与改进,实现一个更先进、更完善的网络嗅探器。
\begin{enumerate}
    \item 增加更多协议的解析支持,如HTTP、FTP、DNS等应用层协议
    \item 优化数据包处理和界面刷新机制,提高程序性能
    \item 增加更强大的数据分析功能,如流量统计图表、异常检测等
    \item 完善跨平台兼容性,确保在不同操作系统上都能稳定运行
\end{enumerate}

\section{经验与体会}
通过本项目的开发,深入理解了网络协议的工作原理和数据包结构,掌握了网络嗅探的核心技术。同时,在项目开发过程中,也锻炼了问题分析和解决能力,学会了如何将复杂的功能分解为模块化的组件,提高代码的可维护性和可扩展性。

这次编程实践也让我切身体会到人工智能的发展迅速与实用价值。借助TRAE IDE中的多模型Agent协作,从项目框架搭建、代码调试到UI界面迭代与文档生成,AI工具有效降低了开发门槛、提升了问题解决效率,让我能够更聚焦于核心功能的逻辑实现与性能优化。这种人机协同的开发模式不仅是技术实践的助力,更启发了我对未来软件开发流程的思考——合理运用AI工具赋能编程工作,既能弥补个体经验的局限,也能为创新实践注入更多可能性,这将成为我后续学习与开发中持续探索的方向。
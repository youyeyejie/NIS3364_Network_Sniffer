\chapter{人工智能协作}

在本次编程实践中,我借助了TRAE IDE 3.0.1的帮助,使用大模型Agent与对话流协助完成工作,切身体会到了Vibe Coding的魅力。

\section{模型列表}
在项目开发流程中,我使用TRAE IDE的Agent并令其自动选择模型。其内置的模型如下:\par
\vspace{1em}
\begin{itemize}
    \item Doubao-Seed-Code
    \item Kimi-K2-0905
    \item GLM-4.6
    \item MiniMax-M2
    \item DeepSeek-V3.1-Terminus
    \item Qwen-3-Coder
\end{itemize}

\section{模型交互}
首先,我使用以下Prompt令大模型为我生成一个基本的项目框架:
\begin{figure}[H]
    \centering
    \includegraphics[width=0.62\linewidth]{graph/agent1.png}
    \caption{基本任务描述}
\end{figure}

接着,我阅读并修改代码,完善详细功能,在运行时遇到报错或需要对UI界面进行修改时询问大模型。最后借助大模型的帮助为整个项目生成一份\texttt{README}文档。

\begin{figure}[H]
    \centering
    % 第一行
    \begin{subfigure}[b]{0.47\linewidth}
        \centering
        \includegraphics[width=\linewidth]{graph/agent2.png}
        \caption{循环导入DEBUG}
        \label{subfig:agent2}
    \end{subfigure}
    \hfill % 两列之间的间距
    \begin{subfigure}[b]{0.47\linewidth}
        \centering
        \includegraphics[width=\linewidth]{graph/agent4.png}
        \caption{生成调试脚本}
        \label{subfig:agent4}
    \end{subfigure}
    
    % 第二行
    \vspace{1em} % 行间距
    \begin{subfigure}[b]{0.47\linewidth}
        \centering
        \includegraphics[width=\linewidth]{graph/agent5.png}
        \caption{重组功能DEBUG}
        \label{subfig:agent5}
    \end{subfigure}
    \hfill
    \begin{subfigure}[b]{0.47\linewidth}
        \centering
        \includegraphics[width=\linewidth]{graph/agent6.png}
        \caption{顶部UI修改}
        \label{subfig:agent6}
    \end{subfigure}
    
    % 第三行
    \vspace{1em}
    \begin{subfigure}[b]{0.47\linewidth}
        \centering
        \includegraphics[width=\linewidth]{graph/agent7.png}
        \caption{重组UI修改}
        \label{subfig:agent7}
    \end{subfigure}
    \hfill
    \begin{subfigure}[b]{0.47\linewidth}
        \centering
        \includegraphics[width=\linewidth]{graph/agent10.png}
        \caption{生成README}
        \label{subfig:agent10}
    \end{subfigure}
    
    \caption{各功能模块调试与修改结果} % 总标题
    \label{fig:all_agents}
\end{figure}
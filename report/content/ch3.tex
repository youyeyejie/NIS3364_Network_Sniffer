\chapter{项目功能展示}

\section{主要功能列表}
\begin{enumerate}
    \item \textbf{实时数据包捕获}:支持在不同网卡上实时捕获网络数据包
    \item \textbf{多协议解析}:支持TCP、UDP、ICMP、IPv4、IPv6、ARP等常见网络协议的解析
    \item \textbf{数据包过滤}:支持源/目标IP、源/目标端口等条件过滤数据包
    \item \textbf{混杂模式支持}:可开启混杂模式捕获网络中所有数据包
    \item \textbf{分片重组}:重组IP分片数据包,还原完整数据内容
    \item \textbf{PCAP包导出与分析}:支持PCAP格式数据包的导出及导入分析
\end{enumerate}

\section{运行展示}

\subsection{选择网卡接口}
在顶部功能区中,可以从下拉列表中选择要监听的网络接口。

\begin{figure}[H]
    \centering
    \includegraphics[width=0.8\linewidth]{graph/select-interface.png}
    \caption{选择网络接口}
    \label{fig:select_interface}
\end{figure}

\subsection{步骤2:设置过滤条件}
在顶部功能区域选中协议类型进行过滤。

\begin{figure}[H]
    \centering
    \includegraphics[width=0.8\textwidth]{graph/set-filter.png}
    \caption{设置过滤条件}
    \label{fig:set_filter}
\end{figure}

\subsection{配置捕获选项}
在以管理员模式运行时,可以点击按钮开启/关闭混杂模式(非管理员模式下该按钮禁用)。

\begin{figure}[H]
    \centering
    \includegraphics[width=0.8\textwidth]{graph/config-capture.png}
    \caption{配置捕获选项}
    \label{fig:config_capture}
\end{figure}

\subsection{控制捕获过程}
点击控制按钮开始/暂停捕获。也可以随时点击清空按钮清除当前捕获的数据包。

\subsection{筛选数据包}
停止捕获后,可以根据IP地址、端口号等条件对捕获到的数据包进行筛选。

\begin{figure}[H]
    \centering
    \includegraphics[width=0.8\textwidth]{graph/filter-packet.png}
    \caption{筛选数据包}
    \label{fig:filter_packets}
\end{figure}

\subsection{查看数据包详情}
在列表中选择数据包,在下方查看详细信息:左侧部分是对数据包的各层协议分析,依赖scapy对数据包各字段的提取实现;右侧为提取到的数据部分,可以选择以十六进制/ASCII/UTF-8编码展示。

\begin{figure}[H]
    \centering
    \includegraphics[width=0.8\linewidth]{graph/show-detail.png}
    \caption{查看数据包详情}
    \label{fig:show-detail}
\end{figure}

对于分片数据包,可以点击顶部功能区的重组按钮,查看完整数据内容。

\begin{figure}[H]
    \centering
    \includegraphics[width=0.8\linewidth]{graph/reassemble-packet.png}
    \caption{重组数据包}
    \label{fig:reassemble-packet}
\end{figure}
\begin{figure}[H]
    \centering
    \includegraphics[width=0.8\linewidth]{graph/show-reassemble.png}
    \caption{重组数据包展示}
    \label{fig:show-reassemblel}
\end{figure}

\subsection{导出为PCAP文件及从PCAP文件导入}
点击顶部功能区的“保存PCAP”或“加载PCAP”按钮,可以调用系统窗口将捕获的数据包保存为PCAP格式,或从PCAP格式文件中加载数据包并进行分析。



\section{功能测试}
项目提供了测试脚本\texttt{test/test.py},用于验证嗅探器的功能:\par
\vspace{1em}

\begin{enumerate}
    \item \textbf{分片发送长报文}:测试嗅探器的分片重组功能
    \item \textbf{发送图像文件}:测试文件内容传输的捕获
    \item \textbf{多协议测试}:测试TCP、UDP、ICMP、ARP等协议的解析
\end{enumerate}\par
\vspace{1em}

在启动嗅探器后,运行测试脚本,可以使用筛选功能辅助观察:由于测试脚本使用以太网卡向IP 192.168.1.1 发送测试数据,因此设置筛选选项:dst\_ip=192.168.1.1。\par

\subsection{分片重组测试}
首先测试分片数据包重组功能:在主界面中选中“分组”列标识为“1”的数据包,点击右上角重组按钮,打开重组窗口:

\begin{figure}[H]
    \centering
    \includegraphics[width=0.9\textwidth]{graph/test1.png}
    \caption{分片重组测试}
    \label{fig:test1}
\end{figure}

可以观察到数据包成功重组,左侧显示重组结果,包含IP ID,源IP、目标IP、数据总长度、是否完整及缺失分片等字段;同时显示其上层协议的相关信息。右侧是重组数据的展示,使用UTF-8解码后,我们成功得到测试脚本中发送的\texttt{README.md}的内容。


\subsection{文件重组测试}
测试脚本中还发送了一个PNG文件,这里将截获到的数据包重组后,使用ASCII解码,发现数据包头包含“PNG”字段。再使用\texttt{test/HEX2PNG.py}脚本将重组后的十六进制数据转换为PNG格式,我们成功得到测试脚本中发送的\texttt{ico/Sniffer.png}的图片,这也是软件打包为二进制文件时使用的ico。

\begin{figure}[H]
    \centering
    \includegraphics[width=0.5\linewidth]{graph/ico.png}
    \caption{重组后的图标}
    \label{fig:ico}
\end{figure}

\begin{figure}[H]
    \centering
    \includegraphics[width=0.9\linewidth]{graph/test2.png}
    \caption{文件重组测试}
    \label{fig:test2}
\end{figure}


\subsection{Ping测试}
在Windows下执行 \verb|ping oc.sjtu.edu.cn -4|,会使用IPv4协议测试canvas网站的连通性,重复发送四次ICMP回显请求。嗅探器捕获到的数据包如图:

\begin{figure}[H]
    \centering
    \includegraphics[width=0.8\linewidth]{graph/ping_v4.png}
    \caption{Ping IPv4测试}
    \label{fig:ping_v4}
\end{figure}

同理,使用 \verb|ping oc.sjtu.edu.cn -6| 进行IPv6的Ping测试,结果如下:
\begin{figure}[H]
    \centering
    \includegraphics[width=0.8\linewidth]{graph/ping_v6.png}
    \caption{Ping IPv6测试}
    \label{fig:ping_v6}
\end{figure}


\subsection{多协议测试}
最后进行多协议解析测试。测试脚本中分别以ICMP、UDP、TCP、ARP发送测试请求,可以观察到网络嗅探器均能正确解析并数据显示。

\begin{figure}[H]
    \centering
    \includegraphics[width=0.8\linewidth]{graph/test3.png}
    \caption{多协议测试}
    \label{fig:test3}
\end{figure}
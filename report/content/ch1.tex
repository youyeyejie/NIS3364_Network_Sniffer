\chapter{网络嗅探简介}

\section{概念原理}
网络嗅探(Network Sniffing)是一种监视网络流量的技术,通过捕获网络中的数据包并进行分析,以了解网络通信情况。它可以帮助网络管理员诊断网络问题、检测网络攻击和优化网络性能。\par

网络嗅探器工作在OSI模型的数据链路层,通过将网络接口设置为混杂模式(Promiscuous Mode),可以捕获经过该接口的所有数据包,而不仅仅是发送给自身的数据包。捕获到的数据包会被按照不同的协议进行解析,提取出源地址、目的地址、协议类型、数据内容等信息。

\section{应用场景}
\begin{itemize}[label=$\bullet$]
    \item 网络故障排查:通过分析数据包,定位网络连接问题、协议错误等
    \item 网络安全监控:检测异常流量、网络攻击(如ARP欺骗、端口扫描等)
    \item 网络性能分析:统计网络流量、分析带宽使用情况,优化网络资源分配
    \item 协议分析与学习:深入了解各种网络协议的工作原理和数据格式
\end{itemize}

\section{注意事项}
\begin{itemize}[label=$\bullet$]
    \item 权限要求:运行网络嗅探器需要管理员/root权限
    \item 网络安全:在使用混杂模式时,请确保符合相关法律法规,仅用于合法的网络分析
    \item 性能影响:长时间捕获大量数据包可能会影响系统性能
    \item 兼容性:在不同操作系统上可能需要调整部分配置
\end{itemize}
